\documentclass[11pt, oneside]{article} % letter, article, report
\usepackage{geometry}                		% See geometry.pdf to learn the layout options. There are lots.
\geometry{letterpaper}                   		% ... or a4paper or a5paper or ... 
\usepackage{graphicx}				% Use pdf, png, jpg, or eps§ with pdflatex; use eps in DVI mode
								% TeX will automatically convert eps --> pdf in pdflatex		
\usepackage{amssymb}

\title{Comp 261 A04 Report}
\author{Diego Trazzi}
%\date{}							% Activate to display a given date or no date

\begin{document}
\maketitle
This assignment has been rather challenging but I relive I have accomplished most of the requirements. In the following section I will discuss details about the code and list few bugs which I have identified thought the process of testing and coding.
\section{Code implementation}
To implement a parser is necessary to have a set of grammar rules. The more strict and precise is the grammar the easier is to implement the parser. 
For this assignment I have implemented a simple parser with prefix notation.\\
Throughout the process of writing code blocks I have utilised the programs provided to check the validity of the implementation.
%\subsection{}
\section{Challenges}
\begin{itemize}
\item For the initial challenge (up to 95) I have implemented in the parser a map to store local variables declared at runtime, so that I can then check if the variable is been assigned or if is not been initialised and throw a parser fail message..
\item Successively, I have implemented the local scope for variables (up to 100). To achieve this I have added a stack of maps (VarNode, RoborExprNode) so that I could then add a new layer at each occurrence of a block, and remove that layer at the end of a bloc, and rely on the implicit structure of a stack (FIFO). This implementation was then tested on a simple program which check for local and global variables.
\end{itemize}
\end{document}